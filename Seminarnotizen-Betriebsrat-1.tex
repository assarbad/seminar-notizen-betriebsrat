% !TeX encoding = UTF-8
% !TeX spellcheck = de_DE_OLDSPELL
% !TEX TS-program = lualatex
\ifdefined\directlua\else
	\errmessage{LuaTeX is required to typeset this document}
	\csname @@end\expandafter\endcsname
\fi
\documentclass[version=last,paper=A4,fontsize=11pt,DIV=18]{scrartcl}
\usepackage{scrlayer-scrpage}
\usepackage{polyglossia}
\setmainlanguage[spelling=old,variant=german]{german}
\usepackage{libertine}
\usepackage{microtype}
\usepackage{fontspec-luatex}
\usepackage{tabularray}
\usepackage{pdfpages}
\usepackage{csquotes}
\usepackage{fontawesome5}
\usepackage{hyperref}
\usepackage{tikz}

\definecolor{linkblue}{rgb}{0, 0, 127}

\hypersetup{
	bookmarksnumbered=false,
	breaklinks=false,
	colorlinks=true,
	citecolor=black,
	filecolor=black,
	linkcolor=black,
	menucolor=black,
	urlcolor=linkblue,
	pdftitle={Notizen aus dem ifb-Seminar Betriebsrat I im Mai 2024},
	pdfcreator={TeXlive + LuaLaTeX},
	pdfkeywords={Betriebsrat,Betriebsverfassungsgesetz,BetrVG},
	pdflang={de},
	pdfpagemode=UseNone,
	pdfpagelayout=OneColumn,
	pdfstartview=FitH,
	pdfencoding=auto,
	psdextra,
}

\usetikzlibrary{positioning,arrows.meta}

%\pagestyle{empty}
%\renewcommand*{\titlepagestyle}{empty}
\setlength{\parindent}{0pt}
%%\usepackage{lua-visual-debug}

\begin{document}

Diese Notizensammlung ist während der Teilnahme am ifb-Seminar \enquote{Betriebsrat I} entstanden aus Äußerungen der Seminarleiterin zu Fragen oder ihren Anmerkungen.

\section*{Rangordnung der Normen}

\begin{center}
	\includegraphics[width=.9\textwidth]{Rangordnung-der-Normen.pdf}
\end{center}

Siehe auch Wikipedia-Artikel \href{https://de.wikipedia.org/wiki/Normenpyramide_im_Arbeitsrecht}{\faIcon{wikipedia-w}~Normenpyramide im Arbeitsrecht}

\section*{Begriffe und Abkürzungen}

\begin{description}
	\item[AG] Arbeitgeber
	\item[AN] Arbeitnehmer; definiert in \S~5~BetrVG, aber auch implizit im BGB -- Besonderheit im BetrVG: leitende Angestellte werden definiert und gelten nicht als vom BR vertretene AN
	\item[ArbR] Arbeitsrecht; geteilt in \faIcon{long-arrow-alt-right}~IndividualArbR und \faIcon{long-arrow-alt-right}~KollektivArbR
	\item[betriebliche Übung] bezeichnet im Prinzip \enquote{Gewohnheitsrechte}; ein Beispiel wären alljährlich gezahlte 13. Monatsgehälter, welche ohne Klausel daherkommen, daß sich aus der Zahlung kein zukünftiger Anspruch ergibt -- nach ein paar solcher Zahlungen \emph{ohne} Klausel, wird das 13. Monatsgehalt ggf. einklagbar.
	\item[BetrVG] Betriebsverfassungsgesetz, definiert alles rund um die Betriebsratstätigkeit; enthält eigenen \enquote{Arbeitnehmer}-Begriff mit klarer Definition um Trennschärfe von \enquote{Leitenden Angestellten} herzustellen, die vom BR nicht vertreten werden!
	\item[BGB] Bürgerliches Gesetzbuch, bspw. referenziert zu allgemeinen Vertragsthemen
	\item[BR] Betriebsrat
	\item[BRM] Betriebsratsmitglied(er); zu unterscheiden nach ordentlichen Betriebsratsmitgliedern und Ersatzbetriebsratsmitgliedern
	\item[BRV] Betriebsratsvorsitzender
	\item[BV] Betriebsvereinbarung: eine vertragliche Vereinbarung zwischen AG und und BR -- betreffs aller AN im Betrieb -- welche in der Rangordnung über einem AV und unter einem TV angesiedelt ist\\
	\textbf{NB:} keine BV ohne Anwalt ausarbeiten! Vorgelegten BV die vom AG ausgearbeitet wurden immer mit großer Skepsis begegnen.\\
	\textbf{Achtung:} immer einen Anwalt mit Spezialisierung auf KollektivArbR wählen!\\
	\textbf{Auch wichtig:} BVs thematisch klein halten, da bspw. einzeln kündbar.
	\item[EBRM] Ersatzbetriebsratsmitglied(er), werden nach einem definierten Modus und in der Reihenfolge\footnote{nach Wahllisten!} in der sie bei einem größeren BR in den BR eingezogen \emph{wären} vom BRV geladen wenn ein Verhinderungsgrund vorliegt, weshalb ein ordentliches BRM nicht an einer BR-Sitzung teilnehmen kann\\
	\textbf{NB:} hierbei ist \S~15~BetrVG zu beachten (\enquote{Minderheitengeschlecht})!
	\item[GBR] Gesamtbetriebsrat, notwendig (verpflichtend!) wenn ein Unternehmen mehrere Betriebe hat, in welchen sich BR konstituiert haben
	\item[EBR] Europäischer Betriebsrat (nach EU-Recht) kann (freiwillig) zwischen Betriebsräten eines Konzerns innerhalb der EU gegründet werden
	\item[Gewsch] Gewerkschaft
	\item[HausTV] Haustarifvertrag wird zwischen \emph{einem} AG und einer Gewsch abgeschlossen; andere Verträge gelten bspw. branchenweit
	\item[IndividualArbR] Individualarbeitsrecht bezeichnet jene Rechtsbereiche des Arbeitsrechts bei denen es um das individuelle Verhältnis von AG zu AN geht
	\item[KBR] Konzernbetriebsrat (nach dt. Recht!), kann (freiwillig) gegründet werden wenn mehrere Betriebsräte innerhalb von Unternehmen desselben Konzerns existieren
	\item[KollektivArbR] Kollektivarbeitsrecht bezeichnet jene Rechtsbereiche im Arbeitsrecht bei denen es um mehrere AN geht, bspw. das BetrVG
	\item[oBRM] ordentliches BRM in Abgrenzung von EBRM
	\item[Regelungsabsprache] eine Verabredung des BR mit dem AG über die Verfahrensweise in bestimmten Belangen; nicht so bindend wie eine BV, auch kein förmlicher Vertrag im Sinne des BGB
	\item[StGB] Strafgesetzbuch; relevant falls man sich etwas zuschulden kommen läßt, bspw. die Offenlegung von Sachverhalten die der Geheimhaltung unterliegen
	\item[TV] Tarifvertrag, üblicherweise branchenweit ausgehandelt zwischen Arbeitgeberverbänden und Gewerkschaften, vgl. \faIcon{long-arrow-alt-right}~HausTV
	\item[Verhinderungsgrund] wird unterschieden nach \emph{tatsächlichen} -- bspw. Urlaub, Krankheit, usw. -- und \emph{rechtlichem} -- bspw. BRM ist unmittelbar und individuell von dem Beschluß betroffen
	\item[VO] Verordnung
\end{description}

\section*{Arbeitsrecht (ArbR)}

\begin{center}
	\begin{tikzpicture}[>=Stealth,rect/.style={draw=black, 
			rectangle, 
			fill=lightgray,
			fill opacity=0.2,
			text opacity=1,
			minimum width=100pt, 
			minimum height = 50pt, 
			align=center}]
		\node[rect] (a1) {Individualarbeitsrecht};
		\node[rect,right=100pt of a1,thick] (a2) {Kollektivarbeitsrecht\\inkl. BetrVG};
		\path (a1) -- (a2) node[midway,below=60pt,rect] (a3) {Arbeitsrecht};
		\draw[<-,thick,gray] (a1.south)--(a3.west)node[midway,sloped,below]{\tiny einzelner Arbeitnehmer};
		\draw[<-,thick,gray] (a2.south)--(a3.east)node[midway,sloped,below]{\tiny viele Arbeitnehmer};
	\end{tikzpicture}
\end{center}

\section*{Zufällig aufgeschnappte Fakten}

\begin{itemize}
	\item Geschäftsführung \emph{immer} fragen, fragen, fragen! Fragen möglichst in einer Form stellen die wenig Spielraum bei der Antwort läßt.
	\begin{itemize}
		\item AG ist zur wahrheitsgemäßen Beantwortung verpflichtet soweit es die Mitbestimmungsrechte des BR betrifft
		\item Die Würdigung eines rechtlichen Sachverhalts könnte bspw. zu einer schwammigen Antwort führen:
		\begin{itemize}
			\item Wie viele Arbeitnehmer nach \S~60~BetrVG (1) gibt es im Betrieb?
			\item Trifft der \S~60~BetrVG auf unsere Werksstudenten zu?
		\end{itemize}
		Besser:
		\begin{itemize}
			\item Wie viele AN unter 18 Jahren gibt es im Betrieb?
			\item Wie viele AN zwischen 18 und 25 Jahren gibt es im Betrieb?
		\end{itemize}
		\item Fragen immer mit (angemessenen) Fristen zur Beantwortung versehen, bspw. zwei Wochen
	\end{itemize}
	\item Gewerkschaftsgründung ist nicht trivial, insofern kann man auch nicht \enquote{einfach} eine Gewerkschaft gründen um bspw. einen Tarifvertrag zu erzielen
	\item Gewerkschaften sind nach Branchen organisiert
	\item Tarifverträge handelt normalerweise eine Gewerkschaft mit den Arbeitgeberverbänden aus, während der Haustarifvertrag zwischen einer Gewerkschaft und einem Arbeitgeber geschlossen wird\footnote{Beispiele sind große Firmen wie die \textsc{Deutsche Bahn}, Autobauer usw.}
	\item Generell gilt der Gleichbehandlungsgrundsatz, es sei denn \emph{sachliche Gründe} stehen dagegen
	\item Bei einem Personalgespräch -- ein definierter Begriff -- kann \emph{immer} ein BRM dabei sein, wobei der betroffene AN diesen hinzuziehen muß!
	\begin{itemize}
		\item Freie Wahl des BRM
		\item Ggf. ein \enquote{Gespräch} bei dem klar wird daß es zu einem Personalgespräch ausartet -- auch AN-seitig!!! -- abbrechen um in einem Folgetermin unter Anwesenheit eines BRM zu vereinbaren
	\end{itemize}
	\item Betrieb $\neq$ Unternehmen, BR organisieren sich auf Betriebsebene, im Unternehmen \emph{muß} es einen GBR geben wenn in mehreren Betrieben des Unternehmens ein BR existiert\footnote{Begründung ist grob gesagt, daß damit BV zu gleichen Sachverhalten nicht einzeln ausverhandelt werden müssen}
	\begin{itemize}
		\item \href{https://www.northdata.com}{Suche über europäische Firmendaten: northdata.com}
		\item \href{https://www.handelsregister.de}{Firmendaten bereitgestellt durch die Bundesländer: handelsregister.de}
	\end{itemize}
	\item Ein Aufhebungsvertrag geht am BR vorbei, auch kein Mitbestimmungsrecht seitens BR
	\item Bei technischen Werkzeugen die AAN betreffend besteht quasi immer ein Mitbestimmungsrecht
	\item Ausgestaltung der Kostenübernahme für den Sachverständigen (\enquote{Anwalt}) -- bspw. bei der Ausarbeitung einer BV -- ist individuell zwischen BR und AG vereinbart. Einige AG geben quasi Blankoscheck, andere fordern Bezug zu einer bestimmten BV.
	\item Die Einigungsstelle anzurufen kann als Druckmittel genutzt werden.
	\item Eine Vergütungsordnung wäre eine BV. Sie ersetzt keinen Tarifvertrag, aber sie ermöglicht innerbetriebliche Absprachen zwischen AN, vertreten durch den BR, und AG.
	\item Der Personal- und Betriebsausschuß haben u.U. das Recht auf Einblick in die Bruttoentgeltlisten (\enquote{Gehaltslisten})
	\item Anwesenheitslisten sollten bspw. häufig geladene EBRM direkt mit auflisten. Bei Verhinderung jeweils ein \enquote{v} für \enquote{verhindert} eintragen, ansonsten Teilnehmer jeweils unterschreiben lassen.
	\begin{itemize}
		\item Anwesenheitslisten geben mittelbar Auskunft über die Beschlußfähigkeit, weshalb die Anfangs- und Endzeit der Sitzung enthalten sein sollte, bzw. bei teilweiser Abwesenheit auch auf die BRM heruntergebrochen.
	\end{itemize}
	\item Es ist klar getrennt welche Sachverhalte zur Einigungsstelle und welche zum Gericht geht -- auch Arbeitsrichter werden zunächst eine außergerichtliche Einigung anstreben (\enquote{Gütetermin})
	\item BV kann es auf allen BR-Ebenen -- EBR, KBR, GBR, BR -- geben
	\item Klausurtagungen des BR können eine geeignete Maßnahme sein um Prioritäten zu klären usw.
	\item Im Konzern gibt es keinen \enquote{guten} und \enquote{bösen} AG. AG ist AG und es spielt keine Rolle ob die \enquote{Ansagen} aus dem Konzern auf Betriebsebene durchgereicht werden.
\end{itemize}

\end{document}
